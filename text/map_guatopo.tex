\subsubsection{El Parque Nacional Guatopo}

Para visualizar la información es recomendable usar
\href{\%22https://nbviewer.jupyter.org/github/esglobe/gee-manual/blob/master/catalogue/map_guatopo.ipynb\%22}{Jupyter
nbview}

\begin{Shaded}
\begin{Highlighting}[]
\CommentTok{# Módulos}
\ImportTok{import}\NormalTok{ ee}
\ImportTok{import}\NormalTok{ folium}
\ImportTok{import}\NormalTok{ geehydro}
\end{Highlighting}
\end{Shaded}

\begin{Shaded}
\begin{Highlighting}[]
\CommentTok{# Inicio}
\NormalTok{ee.Initialize()}

\CommentTok{# Variables iniciales}
\NormalTok{polygonCollection }\OperatorTok{=} \StringTok{'WCMC/WDPA/current/polygons'} \CommentTok{# Coleccion}
\NormalTok{polygonName }\OperatorTok{=} \StringTok{'Guatopo'} \CommentTok{# Imagen}
\end{Highlighting}
\end{Shaded}

\begin{Shaded}
\begin{Highlighting}[]
\CommentTok{# Consultando la coleccion y seleccionando la  imagen}
\NormalTok{polygon }\OperatorTok{=}\NormalTok{ ee.FeatureCollection(polygonCollection) }\OperatorTok{\textbackslash{}}
\NormalTok{            .}\BuiltInTok{filter}\NormalTok{(ee.Filter.eq(}\StringTok{'NAME'}\NormalTok{, polygonName))}
\end{Highlighting}
\end{Shaded}

\begin{Shaded}
\begin{Highlighting}[]
\CommentTok{# Mapa}
\NormalTok{point }\OperatorTok{=}\NormalTok{ [}\FloatTok{10.5165848}\NormalTok{,}\OperatorTok{-}\FloatTok{66.8600685}\NormalTok{] }\CommentTok{# Centro del mapa.}
\NormalTok{Map }\OperatorTok{=}\NormalTok{ folium.Map(location}\OperatorTok{=}\NormalTok{point, zoom_start}\OperatorTok{=}\DecValTok{9}\NormalTok{, control_scale }\OperatorTok{=} \VariableTok{True}\NormalTok{,width}\OperatorTok{=}\StringTok{'100%'}\NormalTok{, height}\OperatorTok{=}\StringTok{'100%'}\NormalTok{,)}
\NormalTok{Map.addLayer(polygon, name}\OperatorTok{=}\NormalTok{polygonName) }\CommentTok{# Agregando poligono}
\NormalTok{Map.setControlVisibility(layerControl}\OperatorTok{=}\VariableTok{True}\NormalTok{, fullscreenControl}\OperatorTok{=}\VariableTok{True}\NormalTok{, latLngPopup}\OperatorTok{=}\VariableTok{True}\NormalTok{) }\CommentTok{# Layer control}
\NormalTok{Map}
\end{Highlighting}
\end{Shaded}

{Make this Notebook Trusted to load map: File -\textgreater{} Trust
Notebook}

Para Guardar el mapa en formato .HTML:

\begin{quote}
Map.save(`mapa\_guatopo.html')
\end{quote}
